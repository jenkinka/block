\section{Project 3: Encrypted Ramdisk}

Jordan Bayles Corey Eckelman Kai Jenkins-Rathburn Jennifer Wolfe

\subsection{Plan}

We started by basing our code off of the sbull.c driver from Linux
Driver Development 3rd Edition. We felt this was a good jumping off
point and that it was very well documented so we would be able to more
easily make the necessary changes to both port it to the current version
of linux, and implement crypto. Additionally, we felt that there would
present enough of a challenge that we would need to understand how
everything works.

We implemented the cryptography functions using the linux crypto API
(linux/crypto.h). The module takes in the key as a module parameter, and
encrypts blocks using it.

We started work on this project by distributing the workload evenly
between ourselves, having two people work on code, one on the writeup
and one on testing. This allowed each of us to share the work and still
each have a good understanding of how the driver is implemented.

\subsection{Implementation}

The driver itself is implemented as a kernel module. It register itself
as a block device, adding itself In each devices setup function it adds
itself as a disk. It stores some specific module parameters including
the amount of devices to store and the request mode.

The block\_device\_operations struct implements .getgeo, as well as many
operations related to acquiring handles or changing media in the drive.

\textbf{\emph{Ramdisk Implementation}} The driver exposes itself as a
block device, which reads and writes to a block of memory, allocated
when it is initialized. We have three different request modes for our
file system: * RM\_NO\_QUEUE: This implements requests using no queue,
and blk\_queue\_make\_request(). * RM\_FULL: This implements requests
using blk\_init\_queue() and a request function that groups requests. *
RM\_SIMPLE: This implements requests using blk\_init\_queue() and a
function that does not group requests.

\textbf{\emph{Crypto Implementation}} In the devices setup function we
call crypto\_alloc\_cipher(), to create a new cipher object, and we free
it in the \emph{exit function with crypto}free\_cipher(). The main bulk
of the crypto work happens in ramdisk\_transfer()- we set our key to the
one defined in the module parameter, and then we either read or write
whole blocks using crypto\_cipher\_decrypt\_one() or
crypto\_cipher\_encrypt\_one() respectively.

\subsection{Implementation Details}

\textbf{\emph{osu\_ramdisk\_init()}} This function registers the ramdisk
using register\_blkdev(), initializes crypto stuff, and allocates teh
ramdisks using setup\_device().

\textbf{\emph{osu\_ramdisk\_exit()}} This function frees each block
device (vfree()s data), then calls unregister\_blkdev(),
crypto\_free\_cipher(), and kfree() in order to clean up.

\textbf{\emph{setup\_device()}} In this function we set up a single
device, creating an osu\_ramdisk\_device instance (e.g.
/dev/osuramdiska). We first zero the memory, and set up its size. Then
we register the correct request function for the device using either
blk\_init\_queue(), or blk\_queue\_make\_request().

\textbf{\emph{osu\_ramdisk\_getgeo()}} We make up what our device looks
like, because it does not physically exist, getting the size in (units)
from the block\_device param passed in and then we set the cylinders
relative to the size.

\textbf{\emph{osu\_ramdisk\_make\_request()}} This function takes a
request from the request queue and processes it using
osu\_ramdisk\_bio\_transfer().

\textbf{\emph{osu\_ramdisk\_full\_request()}} Groups requests, and
handles them similar to osu\_ramdisk\_request().

\textbf{\emph{osu\_ramdisk\_transfer\_request()}} Enumerates over each
request in a bio struct, and calls osu\_ramdisk\_bio\_transfer() on
each. Then, returns the number of sectors written.

\textbf{\emph{osu\_ramdisk\_bio\_transfer()}} Iterates over each
bio\_vec in a given bio struct, and transfers them using
osu\_ramdisk\_transfer().

\textbf{\emph{osu\_ramdisk\_request()}} Handles simple requests.
Transfers each request in the queue by enumerating over it using
blk\_fetch\_request(), and calling osu\_ramdisk\_transfer() on each
request.

\textbf{\emph{osu\_ramdisk\_transfer()}} Transfers data from a buffer to
the disk, encrypting it, or transfers it from the ramdisk to a buffer,
decrypting it.

\subsection{Testing}

-We can test basic functionality by doing reads and writes to the
filesys. -We can test the encryption by printk-ing the encrypted /
decrypte stuff.

\section{Source Code}
